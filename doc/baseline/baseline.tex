\documentclass[12pt,preprint]{aastex}
\usepackage{url}
\usepackage{natbib}
\usepackage{graphicx}
%\usepackage{subfig}
%\usepackage{fixltx2e}

%%%%%%%%%%%%%%%%%%%%%%%%%%%%%%%%%%%%%%%%%%%%%%%%%%%%
%%% author-defined commands
\newcommand\x         {\hbox{$\times$}}
\def\mic              {\hbox{$\mu{\rm m}$}}
\def\about            {\hbox{$\sim$}}
\def\Mo               {\hbox{$M_{\odot}$}}
\def\Lo               {\hbox{$L_{\odot}$}}

\captionsetup[figure]{labelformat=simple}
%%%%%%%%%%%%%%%%%%%%%%%%%%%%%%%%%%%%%%%%%%%%%%%%%%%%


\begin{document}

\title{Moving Object Pipeline Requirements}

\author{}

\begin{abstract}
The Moving Object Processing Software (MOPS) will identify moving
objects observed by LSST, by linking together detections of moving
objects in catalogs created from difference images obtained from LSST
Data Management image processing. Candidate moving objects from each
night will be identified by linking sources moving with linear
velocities compatible with solar system objects (SSOs) to form
`tracklets'. These tracklets will then be linked with candidate SSO
tracklets on nearby nights, where the overall motion can be described
by a quadratic in $RA$ and $Dec$ appropriate for SSOs, to form a
`track'.  Orbits are then fit for the candidate SSO tracks; orbits
with appropriately good fits are considered successful and the SSO
`identified'.  This document provides an overview of the MOPS
algorithms and describes metrics for evaluating the performance of
MOPS.
\end{abstract}

\tableofcontents

%\section{Science Requirements}

\section{Introduction}

Maybe a few words about the point of MOPS and its place in LSST here
(just a paragraph or two). 
 
\section{Overview of MOPS Software}


How MOPS works  - tracklets - tracks - chisq - orbit fitting. 

Precovery and attribution. 


\section{MOPS Metrics \& Scaling}


\section{Development Plan}



\bibliographystyle{apj}
\bibliography{baseline}

\end{document}