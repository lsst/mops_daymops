\documentclass[12pt,preprint]{aastex}
\usepackage{url}
\usepackage{natbib}
\usepackage{graphicx}
%\usepackage{subfig}
%\usepackage{fixltx2e}

%%%%%%%%%%%%%%%%%%%%%%%%%%%%%%%%%%%%%%%%%%%%%%%%%%%%
%%% author-defined commands
\newcommand\x         {\hbox{$\times$}}
\def\mic              {\hbox{$\mu{\rm m}$}}
\def\about            {\hbox{$\sim$}}
\def\Mo               {\hbox{$M_{\odot}$}}
\def\Lo               {\hbox{$L_{\odot}$}}

%\captionsetup[figure]{labelformat=simple}
%%%%%%%%%%%%%%%%%%%%%%%%%%%%%%%%%%%%%%%%%%%%%%%%%%%%


\begin{document}

\title{Moving Object Pipeline Requirements}

\author{}

\begin{abstract}

The Moving Object Processing Software (MOPS) will identify moving
objects observed by LSST, by linking together detections of moving
objects in catalogs created from difference images obtained from LSST
Data Management image processing. Candidate moving objects from each
night will be identified by linking sources moving with linear
velocities compatible with solar system objects (SSOs) to form
`tracklets'. These tracklets will then be linked with candidate SSO
tracklets on nearby nights, where the overall motion can be described
by a quadratic in RA and Dec appropriate for SSOs, to form a
`track'.  Orbits are then fit for the candidate SSO tracks; orbits
with appropriately good fits are considered successful and the SSO
`identified'.  This document provides an overview of the MOPS
algorithms and describes metrics for evaluating the performance of
MOPS.

\end{abstract}

\tableofcontents

%\section{Science Requirements}

\section{Introduction}

The Moving Object Pipeline System (MOPS) is responsible for
discovering solar system objects (SSOs) and identifying their
detections.  SSO discovery refers to the process of analyzing
unidentified sources in the data and grouping those which could be
attributable to an object and identifying the orbit of that object.
SSO identification refers the task of predicting the locations of
known objects and identifying their detections in incoming and legacy
data.

 
\section{Overview of MOPS Software}

The MOPS is based on operating in two different regimes: in
orbit-space, operating on models of objects orbiting the sun; and
sky-plane space, operating on models of apparent motion on the sky.

The process of SSO discovery requires the generation of sets of
detections which may be attributable to the same SSO.  Detections
which belong to the same set are said to be ``linked.'' Initially,
linkages are proposed and filtered using sky-plane models of motion.
Once a set of detections is sufficiently large, MOPS will attempt to
calculate an initial orbit which could be associated with the
specified detections.  If a good orbit is found, then the detections
are associated with the orbit and the system saves the linkage and the
orbit and declares this to be a newly-discovered SSO.  If no such
orbit can be found, the linkage is determined to be a mislinkage and
discarded.  




\section{MOPS Metrics \& Scaling}


\section{Development Plan}



\bibliographystyle{apj}
\bibliography{baseline}

\end{document}
